\chapter{Conclusion and Future}
\label{ch:conclusion}
While our custom inDrop and \acrshort{dropatac} framework did not immediately produce the single-cell data we envisioned, our preliminary results form a strong foundation for future experiments. We were successful in producing functional \acrlongpl{bhb}, a droplet microfluidic cell encapsulation system and both a working \acrshort{rna-seq} and \acrshort{atac-seq} reaction in droplets. In terms of bulk assay properties, both our Drop-seq/inDrop hybrid and the \acrshort{dropatac} prototype showed desirable qualities. Furthermore, we have a number of strong indications that additional fine-tuning of the cell barcodes and sequencing process will bring us to a completely functional \acrshort{scrna-seq} and/or \acrshort{scatac-seq} platform.\pms

% Due to the modular make-up of our protocol and the flexibility of \acrshort{pdms} microfluidics, we are able to rapidly push new prototypes and adapt to different situations.

The next step is to redesign the barcoding process, most likely rolling back the dual-read system to a single-read process or opting for the Illumina TruSeq approach. On the physical level, we want to introduce disulfide linkers in the hydrogel matrix itself in order to dissolve the bead and thereby increase diffusion within the droplet. Furthermore, there is a clear need for optimisation of the single-cell suspension to prevent cell aggregation, which impedes true single-cell measurements. Lastly, we aim to automate several of the steps in the protocol using automated liquid handlers in the future.\pms

In this work, we started on the physical level by producing well-defined \acrlongpl{bhb} and setting up a microfluidic system, transitioned into the molecular-biological sphere when optimising our \acrshort{rna-seq} and \acrshort{atac-seq} assays, and finally ended in the computational domain when analysing the resulting sequencing data. In a span of just 8 months, we were able to set up a complete microfluidic/molecular workflow, showing that Whitesides' original projection has aged well. We have now arrived at a point where graduate students can, under close and careful supervision, independently produce and operate microfluidic systems and put them to use in a rapidly advancing field such as single-cell technology. The field advances rapidly indeed - shortly before this work was submitted, two new droplet-based \acrshort{atac-seq} techniques were published by the Chang and Buenrostro labs \citep{satpathy2019,lareau2019}. Both techniques are being pushed to the market as ready to use packages, showing the translation potential of single cell technology.\pms

The demand for and application potential of single-cell technology is high, but it will take more innovation and large-scale co-operation to get there. With the experience and knowledge gained in this project, our lab is now well-equipped to be part of this effort and help advance single-cell technology to its full potential. Once the barcoding issue is solved, we will look forward to combine \acrshort{scrna-seq} and \acrshort{scatac-seq} into a single droplet-based assay. This hypothetical multi-omic technique could find direct application in large-scale \acrshort{crispr} perturbation screening assays, where the \acrshort{rna-seq} modality can retrieve the exact nature of the \acrshort{crispr} gene edit, and the \acrshort{atac} modality can provide valuable information on that edit's impact on gene regulation. Developing such a combined, multi-modal technique poses several technical challenges, notably concerning the preservation of cellular identity throughout the multi-step protocol that will most likely be required.\pms
