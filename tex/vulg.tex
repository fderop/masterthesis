\renewcommand{\thechapter}{}

\titleformat{\chapter}[hang] % format chapter
{\Huge\mdseries\raggedright}
{}
{0pt}{\Huge\mdseries}

\chapter*{Vulgariserende Samenvatting}
Wat veroorzaakt kanker? Hoe werken onze hersenen? Hoe wordt een embryo gevormd, en waar kan het fout lopen? Deze centrale vraagstukken in de biologie kunnen niet beantwoord worden via de verouderde analysetechnieken van de vorige decennia. Onderzoekers namen toen een weefselstaal, behandelden het met complexe technieken en analyseerden alle resultaten tezamen. Deze aanpak leidt tot een enorm verlies aan informatie, want weefsels bestaan in werkelijkheid uit miljoenen individuele cellen, elk met hun eigen rol. Met onze oude technologie kan geen onderscheid gemaakt worden tussen de verschillende cellen. Vandaag hebben een aantal technologische sprongen het echter mogelijk gemaakt om duizenden cellen individueel te behandelen, en dit binnen een redelijk tijdsbestek en aan een betaalbare prijs - een onmogelijke taak meer dan tien jaar geleden.\pms

Er blijft echter veel plaats voor vordering. De "single-cell" technieken van vandaag staan nog niet op punt. Ze vereisen grote hoeveelheden manueel werk en zijn nog steeds te duur om op routinebasis uitgevoerd te worden. In een nieuwe klasse van single-cell technieken worden individuele cellen opgesloten in een druppel 1 miljoenste van een milliliter groot. De cellen worden elk in hun eigen druppeltje opgelost, en hun inhoud wordt individueel verwerkt. Doordat deze druppels zo klein zijn, kunnen we er vele duizenden tegelijk maken, en hoeven we niet te veel geld te spenderen aan dure reagentia. In dit project combineerden we twee reeds bestaande druppelgebaseerde technieken, wat leidde tot een verbeterde methode. Ook herschreven we een andere methode reeds bestaande methode zodat deze ook in druppels uitgevoerd kan worden.\pms

Data gegenereerd met deze technieken kan ons helpen om processen te begrijpen waar de verschillende rollen van verschillende celtypes belangrijk is, en zal ons een stap dichter brengen bij he antwoord op de grote onopgeloste vragen in de biologie. Bovendien zijn we nu, Door de expertise en kennis opgedaan in dit project, goed uitgerust om aan de frontlijn van het "single-cell" onderziek te staan. De toekomst bevat een aantal spannende uitdagingen, van de ontwikkeling van volledig nieuwe technieken tot het combineren van reeds bestaande methodes. Door direct mee te werken aan het oplossen van deze uitdagingen hopen we single-cell technologie verder te brengen dan wat vandaag mogelijk is, en om zo mee de grenzen van de biologie te verleggen.\pms
