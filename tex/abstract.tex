{\Huge\mdseries\raggedright Summary}\pms
\bigskip
\bigskip
What drives cancer? How do our brains work? How is an embryo formed, and where can it go wrong? These questions are at the heart of biology, but due to the limits of the previous decade's technology, have not yet been answered. Researchers were constrained to taking a tissue sample, processing it, and analysing the resulting dataset as originating from a single entity. Such an approach leads to an enormous loss of information, as tissue consist of millions of individual cells, each with their own role. Today, several technological leaps have made it possible to profile the gene expression of thousands of single cells at a reasonable cost within a realistic time frame, an unthinkable feat a decade ago.\pms

However, much remains to be improved. Today's single-cell assays are prone to noise, demand high amounts of labour and are still too expensive for routine use. This project focuses on improvement in the technological aspect of the single-cell landscape. The first part of this thesis describes how we combined elements of inDrop and Drop-seq, two droplet-microfluidic single-cell RNA-sequencing platforms, into a new protocol. Several new additions and features, both on the physical and the chemical/molecular scale, were tested and refined, producing a final sequencing library which we then analysed (and troubleshooted) computationally. In the second part, we implemented the assay for transposase-accessible chromatin (ATAC-seq) on our custom microfluidic platform to produce a droplet-based single-cell ATAC-seq protocol, a protocol which does not exist in open-source form yet.\pms

While a number of technical limitations prohibited us from efficiently separating our data to the single-cell level, a problem which is examined in great detail in the third and final part of this work, the positive bulk characteristics of the data encouraged us to continue optimisation and refinement of our method. We have since diagnosed part of the problem, and are ready to move forward to an improved prototype. More importantly, due to the experience and knowledge we gained here, we are now well-equipped to be on the front line of single-cell technology. The future still holds several exciting challenges - from the development of new single-cell assays, to the integration of spatial information and finally, the combination of the two. By directly contributing to the solution of these challenges, we hope to push single-cell technology beyond what is possible now, and advance (systems) biology as a whole. Data generated with this technology will finally help us understand those processes where cellular heterogeneity is important, and will bring us one step closer to answering biology's great unsolved mysteries.\pms

\clearpage
{\Huge\mdseries\raggedright Samenvatting}\pms
\bigskip
\bigskip

Wat veroorzaakt kanker? Hoe werken onze hersenen? Hoe wordt een embryo gevormd, en waar kan het fout lopen? Deze vraagstukken staan in het centrum van de biologie, maar bleven onbeantwoord door de limieten van technologie uit de afgelopen decennia. Onderzoekers waren beperkt tot het homogeniseren van een weefselstaal, het "in bulk" te behandelen, en de resulterende data te verwerken als komende van een enkele entiteit. Deze aanpak leidt tot een enorm verlies aan informatie, want weefsels bestaan in werkelijkheid uit miljoenen individuele cellen, elk met hun eigen rol. Vandaag hebben een aantal technologische sprongen het mogelijk gemaakt om de genexpressie van duizenden individuele cellen te profileren, en dit binnen een redelijk tijdsbestek en aan een betaalbare prijs - een onmogelijke taak meer dan tien jaar geleden.\pms

Er blijft echter veel plaats voor verbetering. De single-cell technieken van vandaag genereren veel ruis, vereisen grote hoeveelheden manueel werk en zijn nog steeds te duur om op routinebasis uitgevoerd te worden. Het eerste deel van deze thesis beschrijft hoe we elementen van inDrop en Drop-seq, twee druppel-gebaseerd microflu\"idische RNA-sequencing platforms, combineerden in een nieuw, ge\"integreerd protocol. Een aantal nieuwe addities, zowel op het fysische en chemische vlak, werden getest en verfijnd, leidend tot een sequencing dataset die dan computationeel geanalyseerd werd. In het tweede deel implementeerden we het "assay for transposase-accessible chromatin" (ATAC-seq) op het zelfde microfluidische platform om een druppel-gebaseerde single-cell ATAC-seq protocol te ontwikkelen, een techniek die vandaag nog niet in open-source vorm bestaat.\pms

Hoewel een aantal technische limitaties ons verhinderden onze datasets efficient te splitsen tot het single-cell niveau, een probleem dat uitvoerig onderzocht wordt in het derde en laatste deel van de thesis, moedigden de positieve bulk eigenschappen van onze data ons aan om het optimisatie- en verfijningsproces van onze methode verder te zetten. Inmiddels hebben we een deel van de problemen kunnen identificeren, en zijn we klaar om verder te gaan naar een verbeterd prototype. Bovendien zijn we nu, door de expertise en kennis opgedaan in dit project, goed uitgerust om aan de frontlijn van het "single-cell" onderzoek te staan. De toekomst bevat een aantal spannende uitdagingen - van de ontwikkeling van nieuwe single-cell assays, tot de incorporatie van spatiale informatie, en uiteindelijk de combinatie van de twee. Door direct mee te werken aan het oplossen van deze uitdagingen hopen we single-cell technologie verder te brengen dan wat vandaag mogelijk is, en om de grenzen van de (systeem-) biologie te verleggen. Data gegenereerd via single-cell technologie zal ons helpen om processen te begrijpen waar cellulaire heterogeniteit belangrijk is, en zal ons een stap dichter brengen bij de antwoorden op de grote onopgeloste vragen in de biologie.
\pms
